\documentclass[10pt,a4paper]{article}
\usepackage[latin1]{inputenc}
\usepackage{amsmath}
\usepackage{amsfonts}
\usepackage{amssymb}
\usepackage{syntax} % Grammar writer %
\usepackage{listings} % source code writer %
\usepackage{color}

\definecolor{dkgreen}{rgb}{0,0.6,0}
\definecolor{gray}{rgb}{0.5,0.5,0.5}
\definecolor{mauve}{rgb}{0.58,0,0.82}

\lstset{frame=tb,
  language=Java,
  aboveskip=3mm,
  belowskip=3mm,
  showstringspaces=false,
  columns=flexible,
  basicstyle={\small\ttfamily},
  numbers=none,
  numberstyle=\tiny\color{gray},
  keywordstyle=\color{blue},
  commentstyle=\color{dkgreen},
  stringstyle=\color{mauve},
  breaklines=true,
  breakatwhitespace=true
  tabsize=3
}

\lstdefinelanguage{Emma}
{morekeywords={print,if,else,while,def,return,for},
sensitive=false,
morecomment=[l]{\#},
morestring=[b]",
} 

\lstdefinelanguage{rock}
{morekeywords={one,two,three,four,five,six,seven,eight,nine,ten,eleven,twelve,o,clock,rock,around,the,tonight},
sensitive=false,
morecomment=[l]{//},
morecomment=[s]{/*}{*/},
morestring=[b]",
}

\author{Yang Wang}
\title{The Emma Language Reference}
\begin{document}

\section{Introduction}
This reference manual describes the Emma language and serve as an alternative 
format of the language specification.

The Emma language is an self-education and experimental project. 
The main purpose is for the author to study the theory and practice related
to compiler design and implementation. 
The final product is a high level programming language similar to Python in
spirit, but also incorporate features from other languages such as
IDL (Interactive Data Language), C, Java. 

\subsection{Implementation}
The language will be prototyped in Python and implemented in C. 

\section{Lexical analysis}
\subsection{Line structure}


\subsubsection{Physical lines}
A physical line is terminated by a newline character. 

\subsubsection{Logical lines}
A logical line can be a single physical line, multiple physical lines, or
part of a single physical line.

A logical line is terminated by either a newline character or a semicolon.
Termination of a logical line by a newline character is the most 
straightforward choice to end a logical line and this makes the physical 
line to be also a logical line.

Termination by a semicolon is final. The semicolon is supposed to be used
in a single physical line to separate statements. This allows a single 
physical line to contain multiple logical lines. It also provides means
to enter a long script in an interactive session.

Termination by a newline character can be suppressed explicitly by a line 
continuation symbol, ``\textbackslash". This makes it possible to have a 
logical line span across multiple physical lines. 

For simple statements, they can never cross the boundary of a logical line,
i.e., one logical line is also one statement. 
For compound statements, where newline and semicolon are allowed by the 
syntax (e.g. in between statements in a '{}' pair), they may be consisted 
of multiple logical lines. Note even for a compound statement, if it is
outside of the '{}' pair, line termination is still not allowed. These
rules are better clarified with an example of the ``if-else'' compound 
statement. 

\begin{lstlisting}
# A compound if-else statement
if (x == 1) {
    y = 1
} else {
    y = 0
}
\end{lstlisting}

\begin{lstlisting}
// Hello.java
import javax.swing.JApplet;
import java.awt.Graphics;

public class Hello extends JApplet {
    public void paintComponent(Graphics g) {
        g.drawString("Hello, world!", 65, 95);
    }    
}
\end{lstlisting}


\section{Syntactic analysis}
\subsection{grammar}

\setlength{\grammarparsep}{10pt plus 1pt minus 1pt} % increase separation between rules
\setlength{\grammarindent}{12em} % increase separation between LHS/RHS 
\begin{grammar}


<program> ::= (statement)*

<statement> ::= <CR> 
\alt (stmt_list <CR>)


\end{grammar}

\section{Intermediate representation}

\section{Virtual machine}

\section{Standard library}

\end{document}

\documentclass[10pt,a4paper]{article}
\usepackage[latin1]{inputenc}
\usepackage{amsmath}
\usepackage{amsfonts}
\usepackage{amssymb}
\usepackage{syntax} % Grammar writer %
\usepackage{listings} % source code writer %
\usepackage{color}

% ----------------------------------------------------------
% Source code syntax highlight
\lstdefinelanguage{Emma}
{morekeywords={print,if,else,while,def,return,for,continue,break,and,or,not,class},
sensitive=false,
morecomment=[l]{\#},
morestring=[b]",
morestring=[b]',
} 

\definecolor{dkgreen}{rgb}{0,0.6,0}
\definecolor{gray}{rgb}{0.5,0.5,0.5}
\definecolor{mauve}{rgb}{0.58,0,0.82}

\lstset{frame=tb,
  language=Emma,
  aboveskip=3mm,
  belowskip=3mm,
  showstringspaces=false,
  columns=fixed,
  basicstyle={\small\ttfamily},
  numbers=none,
  numberstyle=\tiny\color{gray},
  keywordstyle=\color{blue},
  commentstyle=\color{dkgreen},
  stringstyle=\color{mauve},
  breaklines=true,
  breakatwhitespace=true
  tabsize=3
}
% End of syntax highlight
% ----------------------------------------------------------

\author{Yang Wang}
\title{The Emma Language Reference}
\begin{document}

\section{Introduction}
This reference manual describes the Emma language and also serves as an  
language specification and development guidance.

The Emma language is an self-education and experimental project. 
The main purpose is for the author to study the theory and best practice related
to compiler design and implementation. 
The final product is envisioned to be a high level programming language similar
to Python in spirit, but also incorporate features from other languages such as
IDL (Interactive Data Language), C, Java. 

The basic target of the language is to be Turning complete. The ultimate
target is to be self-hosting, i.e.\ be able to compile itself.
The current development scope of the language is listed as follows:

\begin{itemize}
\item Functions (with recursive calls)
\item Array support
\item Basic I/O support
\item Object oriented programming support by Class definitions
\item Error handling and interactive debugging similar to IDL
\end{itemize}


\subsection{Implementation}
The language will be prototyped in Python and implemented in C. 
An interactive environment will be provided as well as a batch
run mode. Any programs written in the language will be compiled
to bytecode to boost the execution speed. This means
that a runtime environment of the Emma language is always required to run
the bytecode, since it will not be compiled down to the native machine
code.

\pagebreak

\section{Lexical analysis}
\subsection{Line structure}
An Emma program is consisted of lines of texts. The line structure is
hence one of the basic component of a source program.

\subsubsection{Physical lines}
A physical line is terminated by an end-of-line sequence (EOL). The
sequence is different on different platforms. Unix and alike use ASCII LF
(linefeed), while Windows uses ASCII sequence of CR LF (carriage return followed
by a linefeed). All of these sequences are treated equally as a single 
EOL symbol.

\subsubsection{Logical lines}
A logical line can be a single physical line, multiple physical lines, or
part of a single physical line.

A logical line is terminated by either a EOL or a semicolon.
Termination of a logical line by a EOL is the most 
straightforward choice to end a logical line and this makes a physical 
line to be also a logical line.

A semicolon is supposed to be used in a single physical line to separate 
statements. This allows a single physical line to contain multiple logical
lines, which is useful to enter a long script in an interactive session.

While termination by a semicolon is final, termination by a EOL can be 
suppressed explicitly by a line continuation symbol, ``\textbackslash".
This makes it possible to have a logical line span across multiple 
physical lines. 

Simple statements can never cross the boundaries of a logical line.
For compound statements, the language syntax allows both EOL and semicolon
to be in between statements in a ``\{\}'' pair. Therefore compound statements
may be consisted of multiple logical lines. Note that even for a compound 
statement, line termination is still not allowed outside of the 
``\{\}'' pair. 
These rules are better clarified with the following example of a ``if-else''
compound statement. 

\begin{lstlisting}
# A compound if-else statement span across multiple physical lines
if (x == 1) {
    y = 1;
} else {
    y = 0
}
\end{lstlisting}

Note that both EOL and semicolon can appear inside the ``\{\}'' pairs. 
But lines cannot be terminated outside of the ``\{\}'' pairs. That is
why the ``\{'' has to be in the same line with ``\lstinline$if (x == 1)$''.
Similarly, ``\lstinline$else$'' is in the same line with both the leading
``\}'' and the ending ``\{''.

\subsubsection{Line joining}
The backslash character, ``\textbackslash'', can be used to join two lines 
separated by EOL.
This allows a logical line to span across multiple physical lines as follows:
\begin{lstlisting}
# Line joining by "\" character
x = a + b \
      + c \
      + d
\end{lstlisting}

\subsubsection{Comment}
The hash character, ``\#'', starts a single line comment that runs to the end
of a physical line. There is currently no plan to support multi-line
comment symbols.

\subsubsection{Empty line and whitespace}
Any empty line and whitespace are not part of the Emma language. They are
effectively ignored.

\subsection{Keywords}
Keywords are reserved and listed as follows:
\begin{lstlisting}
print if else while for continue break def return class
\end{lstlisting}

\pagebreak


\section{Syntactic analysis}
\subsection{grammar}

\setlength{\grammarparsep}{10pt plus 1pt minus 1pt} % increase separation between rules
\setlength{\grammarindent}{12em} % increase separation between LHS/RHS 

\begin{grammar}
<program> ::= <statement>*

<statement> ::= "EOL"
	\alt <stmt_list> "EOL"

<stmt_list> ::= (<simple_stmt> \alt <compound_stmt>) (<empty_stmt> (<simple_stmt> \alt <compound_stmt>))* [<empty_stmt>]

<stmt_block> ::= `\{' (<statement>)* <stmt_list> "EOL"* `\}' 
                    \alt `\{' "EOL"* `\}'

<simple_stmt> ::= r_expression
	\alt <assign_stmt>                    
	\alt <print_stmt>
	\alt `continue'
	\alt `break'
	\alt `return' r_expression
	\alt <empty_stmt>
	
<empty_stmt> ::= `;'$^{+}$

<compound_stmt> ::= <if_stmt>
	\alt <while_stmt>
	\alt <for_stmt>
	\alt <def_stmt>
	
<if_stmt> ::= `if' r_expression (<simple_stmt> \alt stmt_block) [<else_stmt>]

<else_stmt> ::= `else' (<if_stmt> \alt <simple_stmt> \alt <stmt_block>)
              
<while_stmt> ::= `while' <r_expression> (<simple_stmt> \alt <stmt_block>)

<for_stmt> ::= `for' "IDENT" `=' <expression>, expression [`,' expression] (<simple_stmt> \alt <stmt_block>)

<def_stmt> ::= `def' <func> <stmt_block>

<print_stmt> ::= `print' [r_expression (`,' <r_expression>)*]

<assign_stmt> ::= ("IDENT" | <slice>) `=' <r_expression>

<func> ::= "IDENT" (<arglist> | <idxlist>)* <arglist>

<arglist> ::= `(' [<r_expression> (`,' <r_expression>)*] `)'

<slice> ::= "IDENT" (<arglist> | <idxlist>)* <idxlist>

<idxlist> ::= `[' [<expression>] (`:' [<expression>])\{0,2\} `]'

<r_expression> ::= <r_term> (<r_orop> <r_term>)*

<r_term> ::= <r_factor> (<r_andop> <r_factor>)*

<r_factor> ::= [<r_unary_op>] <l_expression>

<l_expression> ::= <l_factor> (<l_op> <l_factor>)*

<l_factor> ::= <expression>

<expression> ::= <term> (<addop> <term>)*

<term> ::= <factor> (<mulop> <factor>)*

<factor> ::= [<unary_op>] "NUMBER"
	\alt "STRING"
	\alt "IDENT"
	\alt <func>
	\alt <slice>
	\alt `(' <r_expression> `)'
	
<r_orop> ::= `or' | `xor'

<r_andop> ::= `and'

<l_op> ::= `>' | `<' | `>=' | `<=' | `==' | `!='

<addop> ::= `+' | `-'

<mulop> ::= `*' | `/'

\end{grammar}


\section{Intermediate representation}

\section{Virtual machine}

\section{Standard library}

\end{document}

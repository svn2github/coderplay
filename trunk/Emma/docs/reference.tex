\documentclass[10pt,a4paper]{article}
\usepackage[latin1]{inputenc}
\usepackage{amsmath}
\usepackage{amsfonts}
\usepackage{amssymb}
\author{Yang Wang}
\title{The Emma Language Reference}
\begin{document}

\section{Introduction}
This reference manual describes the Emma language and serve as an alternative 
format of the language specification.

The Emma language is an self-education and experimental project. 
The main purpose is for the author to study the theory and practice related
to compiler design and implementation. 
The final product is a high level programming language similar to Python in
spirit, but also incorporate features from other languages such as
IDL (Interactive Data Language), C. 

\subsection{Implementation}
The language will be prototyped in Python and implemented in C. 

\section{Lexical analysis}
\subsection{Line structure}


\subsubsection{Physical lines}
A physical line is terminated by a newline character. 

\subsubsection{Logical lines}
A logical line can be terminated by either a newline character or a semicolon.
Termination by a semicolon is final. The semicolon is supposed to be used
in a single physical line to separate statements. It also provides a means
to enter a long script in an interactive session.

Termination by a newline character can be suppressed in two ways.
First, an explicit line continuation symbol, ``\textbackslash", concatenates
two lines separated by newline character. Second, 

\section{Synatic analysis}
\subsection{grammar}

\section{Intermediate representation}

\section{Virtual machine}

\section{Standard library}

\end{document}